\chapter{\textit{Thanks}}

\indent 

\textit{Firstly, I would like to thank Professor Daniele Loiacono for his precious help during the months that led to the completion of this work.}

\par \mbox{}

\textit{I would also like to thank my parents, who gave me the opportunity to focus on my studies, my grandmother, who constantly supported me, and my aunt, uncle and cousins from Milan, who often hosted me after a long day of lectures.}

\par \mbox{}

\textit{Finally, I would like to thank my colleague and friend Luca, essential companion in this journey, and all the other friends with whom I shared these intense years.}

\par \mbox{}

\textit{\rightline{Marco Ballabio}}

\chapter{Abstract}

\<Level design> plays a key role in the development of a video game, since it allows to transform the \<game design> in the actual \<gameplay> that the final user is going to experience. Nevertheless, we are still far from a scientific approach to the subject, with a complete lack of a shared terminology and almost no experimental validation for the most used techniques. Even if the video game industry does not acknowledge this problem, in the last years the academic environments have shown an increasing interest towards this subject. \\
We analyzed the main breakthroughs made in level design research applied to the genre of \<First Person Shooters>, devoting particular attention to the ones that try to assist the design process by employing \<Procedural Content Generation>. To support this kind of research, we developed an \<open-source> \<framework> that employs procedural algorithms to generate maps with different topologies, both \<single-level> and \<multi-level>, but that also allows to import maps generated in previous works, thanks to a broad support to the most common export formats used in the literature. The framework was also designed for providing an easy way to define and deploy \<browser-playable online experiments>, that allow to analyze how real users react to different contents. \\
We also explored a novel approach for the analysis of First Person Shooter levels, that uses \<Graph Theory> to extract information about the layout of a map. We used this information to define an approach that uses \<heuristics> to place game elements considering the layout of the map and the features of each element.

\chapter{Sintesi}

Il \<level design> gioca un ruolo chiave nello sviluppo di un videogioco, dal momento che permette di trasformare il \<game design> nell'effettiva esperienza di \<gameplay> che verrà sperimentata dall'utente finale. Nonostante ciò, siamo ancora lontani da un approccio scientifico verso la materia, a causa della completa mancanza di un vocabolario condiviso e della quasi totale assenza di validazione sperimentale per le tecniche più comuni. Anche se l'industria tende ad ignorare questo problema, negli ultimi anni gli ambienti accademici hanno mostrato un crescente interesse verso questo campo. \\
Abbiamo analizzato le principali scoperte fatte nel campo del \<level design> applicato al genere dei \<First Person Shooter>, riservando particolare attenzione ai casi in cui si usa la \<Generazione Procedurali di Contenuti> per assistere il processo di design. Per agevolare questo tipo di ricerca, abbiamo sviluppato un \<framework> \<open-source> che si avvale di algoritmi procedurali per generare mappe con topologie differenti, con uno o più piani, ma che permette anche di importare le mappe generate nei lavori precedenti, grazie ad un vasto supporto per i formati di esportazione più diffusi in questo campo. Il framework è stato anche progettato per consentire la facile creazione di \<esperimenti online giocabili da browser>, che permettono di analizzare come degli utenti reali reagiscono a differenti tipi di contenuto. \\
Abbiamo anche esplorato un nuovo approccio per l'analisi dei livelli per First Person Shooter, che si avvale della \<Teoria dei Grafi> per estrarre informazioni riguardanti il \<layout> di una mappa. Utilizzando queste informazioni, abbiamo definito un approccio basato su \<euristiche> per disporre gli elementi di gioco tenendo conto del layout della mappa e delle caratteristiche di ciascun elemento.
