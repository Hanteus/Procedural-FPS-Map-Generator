\chapter{State of the art}

% INTRODUCTION %

In this chapter we analyze the current state of \<level design> and of its common practices, both in academic and in professional environments, with attention to the genre of \<First Person Shooters> (or \<FPS>).

\par

We then talk about \<Procedural Content Generation> (or \<PCG>), focusing on how it allows to enrich and ease the design process.

\par

Finally, we give an overview of the \<First Person Shooter> genre, analyzing its features, history and evolution, devoting special attention to the games that lead to greater innovation in the field and to the ones that are used to perform academic research in this field.

% LEVEL DESIGN %

\section{Level Design Theory}

\<Level design> is a game development discipline focused on the creation of video game levels.

\par

Today, the \<level designer> is a well-defined and fundamental figure in the development of a game, but it was not always so. In the early days of the video game industry, it was a widespread practice to assign the development of levels to members of the team with other roles, usually programmers. Apart from the limited number of team members and budget, this was because there were no tools such \<level editors>\footnote{\label{levelEditorFootnote}A level editor is a software used to design levels, maps and virtual worlds for a video game. An individual involved with the creation of game levels is a level designer.}, that allowed the \<level designer> to work on a level without being involved with code.

\par

The \<level designer> has a really important role in the development of a good game, since he is responsible for the creation of the world and for how the player interacts with it. The level designer takes and idea, which is the game design, and makes it tangible.
Despite the importance of this role, after all this years, it has not been established a common ground or a set of standards yet, instead, \<level design> is often considered as a form of art, based on heuristics, observation, previous solutions and personal sensibility.

\par

In addition to game play, the game designer must consider the visual appearance of the level and the technological limitations of the \<game engine>\footnote{\label{gameEngineFootnote}A game engine is a software framework designed for the creation and development of video games.}, combining all this elements in a harmonic way.

\par

One of the core components of level design is the \[''level flow'']. For single player games it translates into the series of actions and movements that the player needs to perform to complete the level. A good practice for \<level design> is to guide the player in a transparent way, by directing his attention towards the path he needs to follow. This can be achieved in different ways. Power up and items can be used as breadcrumbs to suggest the right direction in a one way fashion, since they disappear once picked up. Lighting, illumination and distinctly colored objects are another common approach to this problem. A brilliant example of this is \<Mirror's Edge>\footnote{\label{}Digital Illusions CE, 2008}, which uses a really clear color code, with red interactive objects in an otherwise white world, to guide the player through its fast-paced levels. There are also even more inventive solutions, like the dynamic flock of birds in \<Half Life 2>\footnote{\label{}Valve, 2004} used to catch the player attention or to warn him of incoming danger\cite{GuidingThePlayersEye}. Finally, sounds and particular architectures are other elements that designers can use to guide the player. In the academic environment, a lot of researchers have analyzed the effectiveness of this kind of solutions: Alotto\cite{HowLevelDesignersAffect} considers how architecture influences the decisions of the player, whereas Hoeg\cite{TheInvisibleHand} also takes into account the effect of sounds, objects and illumination, with the last being the focus of Brownmiller's\cite{InGameLigthing} work.

\par

In multiplayer games the \<level flow> is defined by how the players interact with each other and with the environment. Because of this, the control of the \<level designer> is less direct and is exercised almost exclusively by modeling the map. Considering \<FPS>, the \<level flow> changes depending on how much an area is attractive for a player. The more an area is easy to navigate or offers tactical advantage, such as cover, resources or high ground, the more players will be comfortable moving in it. This doesn't mean that all areas need to be designed like this, since zones with a ''bad'' flow but an attractive reward, such as a powerful weapon, force the player to evaluate risks and benefits, making the game play more engaging. The conformation of the map and the positioning of interesting resources are used to obtain what Güttler et al.\cite{Guttler:2003:SPL:963900.963915} define as \[''points of collisions''], i.e. zones of the map were the majority of the fights are bound to happen. Moving back to academic research, Güttler et al. have also noticed how aesthetic design loses importance in a multiplayer context. Other researches are instead focused on finding \[patterns] in the design of multiplayer maps: Larsen\cite{LevelDesignPatterns} analyzes three really different multiplayer games, \<Unreal Tournament 2004>\footnote{\label{}Epic Games, 2004}, \<Day of Defeat: Source>\footnote{\label{}Valve, 2005} and \<Battlefield 1942>\footnote{\label{}DICE, 2002}, identifying shared patterns and measuring their effect on gameplay, suggesting some guidelines on how to use them, whereas Hullet and Whitehead identify some patterns for singleplayer games\cite{Hullett:2010:DPF:1822348.1822359}, many of whom are compatible with a multiplayer setting, with Hullett also proving cause-effect relationships for some of this patterns by confronting the hypnotized results with the ones observed on a sample of real players\cite{TheScienceOfLevelDesign}. Despite these experimental results contributing to a formalization of \<level design>, we are still far from a structured scientific approach to the subject.
 
% PROCEDURAL CONTENT GENERATION %

\section{Procedural Content Generation}

% PROCEDURAL CONTENT GENERATION IN FPS %

\section{Procedural Content Generation for First Person Shooter maps}

% FPS DESIGN %

\section{History of Level Design in FPS}

% SUMMARY %

\section{Summary}